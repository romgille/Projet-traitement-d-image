\section{Comparaison avec la photo avec une légère
rotation}\label{comparaison-avec-la-photo-avec-une-luxe9guxe8re-rotation}

\begin{figure}[htbp]
\centering
\includegraphics{../../photos/rotate.jpg}
\caption{Photo rotation légère}
\end{figure}

\begin{table}[htbp]
\centering
\begin{tabular}{llr}
\bfseries Formes &
\bfseries Bhattacharyya (\%)%
\DTLforeach*[\DTLiseq{\fichier}{photos/rotate.jpg}]{valeurs}{%
\fichier=Fichier, \formes=Formes,\bhatta=Bhattacharyya, \hue=Hue, \saturation=Saturation, \value=Value}{%
\\
\formes & \bhatta}
\end{tabular}
\end{table}


Pour cette comparaison, on observe une très faible différence de couleurs
(seulement $0.64 \%$), car l'image est seulement retournée donc les couleurs ne
changent quasiment pas. Par contre la différence de formes est assez élevée car
tous les pixels sont décalés ($53.58 \%$ de différence après application du
filtre de Sobel).\\On constate donc que la rotation légère ne change pas
beaucoup les couleurs de la photo mais majoritairement ses formes.
