\section{Comparaison avec la photo avec effet
poster}\label{comparaison-avec-la-photo-avec-effet-poster}

\begin{figure}[htbp]
\centering
\includegraphics{../../photos/poster.jpg}
\caption{Photo poster}
\end{figure}

\begin{table}[htbp]
\centering
\begin{tabular}{llr}
\bfseries Formes &
\bfseries Bhattacharyya (\%)%
\DTLforeach*[\DTLiseq{\fichier}{photos/poster.jpg}]{valeurs}{%
\fichier=Fichier, \formes=Formes,\bhatta=Bhattacharyya, \hue=Hue, \saturation=Saturation, \value=Value}{%
\\
\formes & \bhatta}
\end{tabular}
\end{table}


La comparaison de cette image nous donne une différence de $16.35 \%$
pour la colorimétrie avec la distance de Bhattacharyya et de $16.79 \%$
pour les formes après application du filtre de Sobel. Ce sont des
pourcentages similaires, cependant selon nos critères, la différence
colorimétrique est trop élevée pour dire que les Images se ressemblent.
Cependant pour la différence de formes via l'application du filtre de
Sobel, le pourcentage est suffisamment faible pour considérer que les
Images se ressemblent d'un point de vue des formes. \\
À l'\oe il nu, on peut observer que les ombres sont vraiment délimités sur la
photo poster, ce qui explique les $16 \%$ de différence trouvés après
application du filtre de Sobel.
